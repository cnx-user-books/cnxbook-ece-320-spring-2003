\documentclass[11pt]{handout}
\usepackage{moreverb}
\usepackage{epsfig}

\renewcommand{\coursetitle}{ECE 320}
\renewcommand{\handouttitle}{Using the Serial Port and \matlab}
\renewcommand{\handoutauthor}{Daniel Sachs and Micheal Kramer}
\renewcommand{\semestertitle}{Fall 1999}

\newcommand{\bea}{\begin{eqnarray}}
\newcommand{\eea}{\end{eqnarray}}

\setlength{\parindent}{5mm}
\begin{document}

\setlength{\baselineskip}{0.5cm}
\setlength{\parskip}{0.5cm}

\makeboxtitle
\vspace{0.3cm}

\section{Using the Serial Port}

%AUTHOR: Daniel Sachs
The core file also supports the serial port installed on the
DSP320C549 DSP. The serial port on the EVM is attached to COM2 on
the PC.  Before jumping to your code, the core file initializes
the EVM's serial port to 38,400 bits per second with no parity, 1 stop bit
and 8 data bits. It then accepts characters received from the PC by
the UART (Universal Asynchonous Receiver/Transmitter) and buffers them
in memory until your code retrieves them. It also can accept a block of
bytes to transmit and send them to the UART in sequence.

Two macros are used to control the serial port: \verb+READSER+ and
\verb+WRITSER+.  Both accept one parameter. \verb+READSER <n>+ reads up
to \verb+<n>+ characters from the serial input buffer (the data coming
from the PC), and places them in memory starting at *AR3. (AR3 is left
pointing one past the last memory location written.) The actual number
of characters read is left in AR1. If AR1 is zero, then no characters
were available in the input buffer.

\verb+WRITSER <n>+ adds \verb+<n>+ characters starting at *AR3 to the
serial output buffer - in other words, it queues them to be sent to the
PC. AR3 is left pointing one location after the last memory location read.

Note that READSER and WRITSER modify registers AR0, AR1, AR2, AR3, and
BK as well as the flag TC. Be sure you restore these registers after
calling \verb+READSER+ and \verb+WRITSER+ if you need them later in
your code.

Note also that the core file allows up to 126 characters to
be stored in the input and output buffers. No checks to protect against
buffer overflows are made, so do not allow the input and output buffers
to overflow. (The length of the buffers can be changed by changing
\verb+ser_rxlen+ and \verb+ser_txlen+ values in the \verb+core.asm+
file.) The buffers are 127 characters long; however, the code
cannot distinguish between a completely-full and completely-empty
buffer. Therefore, only 126 characters can be stored in the buffers.

% add buffer-query macros?

It is easy to check if the input or output buffers in memory are
empty. The input buffer can be checked by comparing the values stored in
the memory locations \verb+srx_head+ and \verb+srx_tail+; if both memory
locations hold the same value, the input buffer is empty. Likewise, the
output buffer can be checked by comparing the values stored in memory
locations \verb+stx_head+ and \verb+stx_tail+. The number of characters in
the buffer can be computed by subtracting the head pointer from the tail
pointer; add the length of the buffer (normally 127) if the resulting
distance is negative.

The following example shows the minimal amount of code necessary to
echo received data back through the serial port. It has been placed in
\verb+V:\54x\dsplib\+ as \verb+ser_echo.asm+.

\setlength{\baselineskip}{0.4cm}
\listinginput{1}{ser_echo.asm}
\setlength{\baselineskip}{0.5cm}

On Line 8, we tell \verb+READSER+ to receive into the location \verb+hold+
by setting AR3 to point at it. On Line 9, we call \verb+READSER 1+ to
read one serial byte into \verb+hold+; the byte is placed in the low-order
bits of the word and the high-order bits are zeroed. If a byte was read,
AR1 will be set to 1. AR1 is checked in Line 12; Line 13 branches back to
the top if no byte was read. Otherwise, we tell reset AR3 to \verb+hold+
(since \verb+READSER+ moved it), then call \verb+WRITSER+ to send the
word we received on Line 16. On Line 18, we branch back to the start to
receive another character.


\section{Using MATLAB to Control the DSP}

%AUTHOR: Michael Kramer
One of MATLAB's features is its ability to quickly create a visual interface
that lets you use standard GUI controls (such as sliders, checkboxes and
radio buttons) to call MATLAB scripts. An example of such an interface is 
available in the \verb+V:\MAT_INT+ directory, which contains two files:
\begin{itemize}
  \item{\verb+ser_set.m+: Initializes the serial port and user interface}
  \item{\verb+wrt_slid.m+: Called when sliders are moved to send new data}
\end{itemize}

\section{Creating a MATLAB user interface}

The following code (\verb+ser_set.m+) initializes the serial port COM2, then
creates a minimal user interface consisting of three sliders.

\setlength{\baselineskip}{0.4cm}
\listinginput{1}{ser_set.m}
\setlength{\baselineskip}{0.5cm}

Line 4 of this code uses the Windows NT \verb+mode+ command to set up
COM port 2 (which is connected to the DSP) to match the serial port 
settings on the DSP evaluation board: 38,400 bps, no parity, 8 data bits
 and 1 stop bit. Line 7 then creates a new MATLAB figure for the controls;
this prevents the controls from being overlaid on to of any graph you
may have already created. 

Lines 12 through the end create the three sliders for the user interface. 
Several parameters are used to specify the behavior of each slider. The
first parameter, \verb+Fig+, tells the slider to create itself in the 
window we created in Line 7. The rest of the parameters are property/value
pairs:

\begin{itemize}
  \item{\verb+units+: \verb+Normal+ tells MATLAB to use positioning relative 
        to the window boundaries.}
  \item{\verb+pos+: Tells MATLAB where to place the control.}
  \item{\verb+style+: Tells MATLAB what type of control to place. \verb+slider+
        creates a slider control.}
  \item{\verb+value+: Tells MATLAB the default value for the control.}
  \item{\verb+max+: Tells MATLAB the maximum value for the control.}
  \item{\verb+min+: Tells MATLAB the minimum value for the control.}
  \item{\verb+callback+: Tells MATLAB what script to call when the control is 
        manipulated. \verb+wrt_slid+ is a MATLAB file that writes the values of
        the controls to the serial port.}
\end{itemize}

\subsection{The User Interface Callback Function}

Every time a slider is moved, the \verb+wrt_slid.m+ file is called:

\setlength{\baselineskip}{0.4cm}
\listinginput{1}{wrt_slid.m}
\setlength{\baselineskip}{0.5cm}

Line 4 of \verb+wrt_slid.m+ opens COM2 for writing. (It's already
been initialized.) Then Line 7 retrieves the value from the slider
using MATLAB's \verb+get+ function to retrieve the \verb+value+
property. The value is then rounded off to create an integer,
and the integer is sent as an 8-bit quantity to the DSP in Line 8.
(The number that is sent at this step will appear when the serial
port is read with \verb+READ_SER+ in your code.) Then the other
two sliders are sent in the same way.

Line 17 sends 0xFF (255) to the DSP, which can be used to indicate
that the three previously-transmitted values represent a complete set
of data points. This can be used to prevent the DSP and MATLAB from
losing synchronization if a transmitted character is not received by
the DSP.

Line 20 closes the serial port. Note that MATLAB buffers the data
being transmitted, and data is often not sent until the serial port
is closed. Make sure you close the port after sending a data block
to the DSP.

\section{Implementation}

Implementation of this lab should be done in two stages. First, use
the new \verb+core.asm+ file and the 6-channel board to reimplement
the variable-delay feedback echo that you wrote for Project Lab 1,
adding in additional test points to take advantage of the 6-channel
output. When this is complete, you will use the serial port on the DSP
EVM to implement a MATLAB GUI that allows the two system gains and the
echo delay to be changed using on-screen sliders. 

Don't forget to set the PMST register to 0xFFE0 before you load code
that has been built using the 6-channel core file.

\subsection{Feedback System Implementation}

\begin{figure}[htb]\centerline  {
\epsfxsize=.6\textwidth\
\epsffile{system.eps}          }
\caption{Feedback System with Test Points}
\label{fig: system}
\end{figure}

First, implement the Figure \ref{fig: system} shown using the 
surround board and its \verb+core.asm+ file. Make
sure to use the \verb+READPROG+ and \verb+WRITPROG+ macros to access
the external program RAM instead of the \verb+reada+ and \verb+writa+
opcodes you used before.  You will use both channels of input by summing
the two inputs (so that either one, or both, may be used as an input to
the system). You will use the multichannel output by sending signals at
several test points to the 6-channel board's D/A converters:

\begin{itemize}
  \item{The summed input signal}
  \item{The input signal after gain stage $G1$}
  \item{The data going into the long delay}
  \item{The data coming out of the delay}
\end{itemize}

Note that the data going into the long delay (D/A 3 in the Figure
\ref{fig: system}) is the output of the system you implemented in Project
Lab I.

As you implement this code, ensure that the delay $n$ and the gain
values $G_1$ and $G_2$ are stored in memory and can be easily changed
using the debugger. If you do this, it will be easier to extend your
code to accept its parameters from MATLAB in Part 2.

\subsection{MATLAB Interface Implementation}

Using the MATLAB interface outlined above as a base, write MATLAB code to send
commands to the serial interface based on three sliders: two gain sliders
(for $G_1$ and $G_2$) and one delay slider (for $n$). Then modify the code
you wrote reimplementing Project Lab I to accept those commands and change
the values for $G_1$, $G_2$ and $n$. Make sure that $n$ can be set to
values spanning the full range of 0 to 131,072, although it is not necessary
that every number in that range be represented. 

\end{document}

