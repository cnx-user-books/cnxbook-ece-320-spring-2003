
%
%
% Module: external_memory_processor_exercise
%
% Author: Daniel Sachs
%
%

You will be implementing three audio effects for this lab: a 2-second delay,
a variable delay, and an feedback echo.

\subsection{Delay Implementation}

For this project lab, you will implement a 131,072 sample
delay using external memory. Use memory locations 010000h - 02ffffh in
extended program RAM to do this; you'll probably also want to use the
\verb+dld+ and \verb+dst+ opcodes to store and retrieve the 32-bit
addresses from the accumulators. Note that these two operations store
the words in memory in big endian order, with the high-order word first.

Use the \verb+reada+ and \verb+writa+
instructions to read and write the extended program memory.
Since the \verb+thru.asm+ file you were given in Lab 1 clears the
OVLY bit, you can assume that you can read or write from all of the
extended program memory without overwriting you program code or
data area.

Remember that arithmetic operations that act on the accumulators, such
as the \verb+add+ instruction, operate the complete 32 or 40 bit wide
value. Also keep in mind that since 131,072 is a power of two, you can
use masking (via the \verb+and+ instruction) to implement the circular
buffer easily.  This delay will be easy to verify on the oscilloscope.

\subsection{Variable delay implementation}

Once you have your delay code working with a 131,072 sample delay,
change it so that the delay can be easily changed to any arbitrary
length between zero (or one) and 131,072 samples by changing the value
stored in one double-word pair in memory. You should keep the buffer
length as 131,072 and change only your addressing of the sample being
read back; it is more difficult to change the buffer size to a length
that is not a power of two.

Verify that your code works as expected by timing the delay from
input to output and ensuring that it is approximately the
right length.

\subsection{Feedback echo implementation}

Last, chnage your code so that the value taken from the end of
the variable delay above is multiplied by a gain factor and
then added back into the input, and the result is both saved
into the delay line and sent out to the digital-to-analog converters.
(It may be necessary to multiply
the input by a gain as well to prevent overflow.) This will make
a one-tap feedback echo, an simple audio effect that sounds remarkably
good.

To test this, hook up the DSP EVM input to the CD player on the
computer, and the output to one of the pairs of speakers in the
lab. This can be done using the RCA to 1/8 inch (headphone jack)
converter plugs available in the lab.
Verify that the echo can be heard multiple times, and that the
spacing between echos matches the delay length you've chosen.
