\documentclass[11pt]{handout}

%\setlength{\oddsidemargin}{-0.25in}
%\setlength{\evensidemargin}{-0.25in}
\setlength{\topmargin}{-0.5in}
%\setlength{\textwidth}{7in}
%\setlength{\textheight}{9in}

\setlength{\parindent}{0in}
\setlength{\parskip}{0.13in}

\renewcommand{\coursetitle}{ECE 320}
\renewcommand{\handouttitle}{An Introduction to TMS320C54x Addressing Modes}
\renewcommand{\handoutauthor}{Mark Haun}
\renewcommand{\semestertitle}{Spring 2000}

\begin{document}

%\begin{center}
%{\LARGE\bf An Introduction to TMS320C54x Addressing Modes}\\[0.1in]
%{\Large ECE320, Spring 2000}
%\end{center}
\makeboxtitle
\vspace{0.3cm}

Microprocessors provide a number of ways to specify the location of data
to be used in calculations.  For example, one of the data values to be
used in an
{\tt add} instruction may be encoded as part of that
instruction's {\it opcode}, the raw machine language produced by the
assembler as it parses your assembly language program.  This is known as
{\it immediate addressing}.  Or, perhaps the opcode will instead contain
a memory address which holds the data ({\it direct addressing}).
More commonly, the instruction will specify that an auxiliary register
holds the memory address which in turn holds the data
({\it indirect addressing}).  The processor knows which addressing mode
is being used by examining special bit fields in the instruction opcode.

Knowing the basic addressing modes of your microprocessor is important
because they map directly into assembly language syntax.  Many annoying
and sometimes hard-to-find bugs are caused by inadvertently using the wrong
addressing mode in an instruction.
%The TMS320C54x assembly language seems
%especially prone to these errors, as the addressing syntax rules often
%appear inconsistent.
Also, in any assembly language, the need to use a
particular addressing mode often dictates which instruction one picks for
a given task.

Chapter five, {\it Data Addressing}, in the {\it TMS320C54x DSP Reference
Set, Volume One:  CPU Reference} contains extended descriptions of most
of the addressing modes described below.


\section{Accumulators: {\it src, dst}}

Whenever the abbreviations {\it src} or {\it dst} are used in the assembly
language syntax description for an instruction, it means that only the
accumulators A and B may be used for that particular operand.  These
are seen everywhere, but two classic examples are {\tt LD}, which always
loads data into an accumulator from somewhere else, and
{\tt STH/STL}, which always store data from an accumulator to somewhere
else.

Examples:

{\tt
\begin{tabular}{@{\hspace{0.4in}}l@{\hspace{0.3in}}l@{\hspace{0.2in}}l}
ld   & *ar5,a  & ; sets A = (contents of memory location pointed to by AR5)\\
sth  & b,*ar7+ & ; sets (contents of memory location pointed to be AR7) = B,\\
     &         & ; \hspace{0.3in} and then increments AR7 by one\\
\end{tabular}
}


\section{Memory-mapped Registers: {\it MMR, MMRx, MMRy}}

Many of the '54x registers are memory-mapped, meaning that they occupy
real addresses at the low end of data memory space.  The most commonly
used of these are the auxiliary registers AR0 through AR7.  Whenever the
abbreviation {\it MMR} is used in the assembly language syntax description
for an instruction, it means that any memory-mapped register may be used
for that particular operand.  Only eight instructions use memory-mapped
register addressing:  {\tt LDM}, {\tt MVDM}, {\tt MVMD}, {\tt MVMM},
{\tt POPM}, {\tt PSHM}, {\tt STLM}, and {\tt STM}.  With {\tt MVMM},
since the instruction accepts two memory-mapped register operands,
{\it MMRx} and {\it MMRy}, only AR0--AR7 and SP may be used.

Don't use a star in front of ARx variables here, since this is not indirect
addressing.

Examples:

{\tt
\begin{tabular}{@{\hspace{0.4in}}l@{\hspace{0.3in}}l@{\hspace{0.2in}}l}
mvmm & ar3,ar5 & ; sets AR5 = AR3\\
stm  & \#5,ar2 & ; sets AR2 = 5\\
ldm  & ar0,a   & ; sets A = AR0\\
\end{tabular}
}


\section{Immediate Addressing: {\it \#k3, \#k5, K, \#k9, \#lk}}

{\it Immediate addressing} means that the numerical value of the data is
itself provided within the assembly instruction.  Various '54x instructions
allow immediate data of 3, 5, 8, 9, or 16 bits in length, which are signified
in the assembly language syntax descriptions with one of the above symbols.
The 16 bit form is the most common and is signified by {\it \#lk}.  16 bit
immediate values always require an extra instruction word and therefore take
an extra machine cycle to execute.

An immediate data operand is almost always specified in assembler syntax
by prepending a pound sign ({\tt \#}) to the data.  Depending on the
context, the assembler may assume that you meant immediate addressing
anyway.

Examples:

{\tt
\begin{tabular}{@{\hspace{0.4in}}l@{\hspace{0.3in}}l@{\hspace{0.2in}}l}
ld   & \#0,a   & ; sets A = 0\\
cmpm & ar1,\#1 & ; sets flag TC = 1 if AR1 == 1; else TC = 0\\
\end{tabular}
}

Labels make this more complicated.  Recall that a label in your assembly
code is nothing more than shorthand for the memory address where the
labeled code or data is stored.  So does an instruction like

{\tt
\begin{tabular}{@{\hspace{0.4in}}l@{\hspace{0.3in}}l@{\hspace{0.2in}}l}
stm  & coef,ar2 & ; sets AR2 = memory address of label coef\\
\end{tabular}
}

mean to store the contents of memory location {\tt coef} in AR2, or does
it mean to store the memory address {\tt coef} itself in AR2?
The second interpretation is correct.  Because the {\tt STM}
instruction has only one form, expecting a {\it \#lk} immediate operand,
the assembler doesn't care whether the label is prefixed with a pound sign
or not.  Still, it would have been better for us to include the pound sign
in the above example for clarity.

Many instructions have several versions allowing the use of different
addressing modes (see {\tt LD} for a good example of this).  With these
instructions, including the pound sign is not optional when specifying
immediate addressing.  The only safe rule, then, is to always prefix
the label with a pound sign if you wish to specify the memory address
of the label and not the contents of that address.

If you are not sure how a particular instruction has been assembled, you
can always examine the {\tt .lst} file produced by the assembler, and compare
the hex opcodes listed to the left of the assembly instructions with the
assembly opcodes given in TI's '54x assembly language manual (Chapter 4
of the {\it Instruction Set Reference}).


\section{Direct Addressing:  {\it Smem} and others}

In the modes called {\it direct addressing} by TI, the instruction opcode
contains a memory offset (see the ``dma'' bits on page 5-8 of the {\it CPU
Reference}) seven bits long which is combined with either the DP (data
pointer) or SP (stack pointer) registers to obtain a complete 16 bit
data memory address.  This divides the data memory into pages of 128 words
each.

SP is initialized for you in the core assembly file and should not need to
be modified.  SP-referenced direct addressing is used by the {\tt PSHD},
{\tt PSHM}, {\tt POPD}, and {\tt POPM} instructions for stack manipulation,
as well as by all subroutine calls and returns, which save program addresses
on the stack.
(Ask a TA if you are not familiar with the function of the stack in a
microprocessor system).

DP-referenced
direct addressing is available wherever you see the {\it Smem} abbreviation
in an assembly syntax description.  The advantage of DP-referenced
addressing over the {\it *(lk)} form described in the next section is that
DP-referenced
addressing will not add an extra instruction word (and corresponding extra
machine cycle).  The disadvantage is that it is limited to 128 words of
contiguous memory, and you have to make sure that DP points to the right
128 words.  DP may be changed with the {\tt LD} instruction as needed.

Examples:

{\tt
\begin{tabular}{@{\hspace{0.4in}}l@{\hspace{0.3in}}l@{\hspace{0.2in}}l}
ld   & 10,a & ; sets A = (contents of memory location DP + 10)\\
add  & 6,b  & ; sets B = B + (contents of memory location DP + 6)\\
\end{tabular}
}

(Make sure you understand that the numbers 10 and 6 above are interpreted
as memory addresses, {\it not} data values.  To get data values, you would
need to use a pound sign in front of the numbers.)


\section{Absolute Addressing: {\it dmad, pmad, *(lk)/Smem}}

This seems to be TI's term for all the forms of direct
addressing which it does not call direct addressing!  It is represented
in assembly instruction syntax definitions using one of the above
abbreviations ({\it *(lk)} addressing is available when the syntax
definition says {\it Smem}).


\subsection{\it dmad}
\vspace{-0.1in}

{\it dmad} (Data Memory ADdress) operands are used by {\tt MV}xx data move
instructions, and represent 16 bit memory addresses in data memory whose
contents are used in the instruction.

Example:

{\tt
\begin{tabular}{@{\hspace{0.4in}}l@{\hspace{0.3in}}l@{\hspace{0.3in}}l@{\hspace{0.2in}}l}
f3ptr & .word & 0          & ; reserve one word of storage; initialize to 0\\
      & \dots\\
      & mvdm  & f3ptr,ar4  & ; set AR4 = memory address of f3ptr\\
\end{tabular}
}


\subsection{\it pmad}
\vspace{-0.1in}

{\it pmad} (Program Memory ADdress) operands are used by the {\tt FIRS},
{\tt MACD}, {\tt MACP},
{\tt MVDP}, and {\tt MVPD} instructions, as well as all subroutine calls
and branching instructions.  They represent 16 bit addresses
in program memory whose contents are used in the instruction, or
jumped to in the case of branch instructions.  Other than subroutine calls
and branches, in this class, the main reason for using a {\it pmad} would
be for the {\tt FIRS} instruction.

Example:

{\tt
\begin{tabular}{@{\hspace{0.4in}}l@{\hspace{0.3in}}l}
firs & *ar3+,*ar4+,coefs\\
\end{tabular}
}

({\tt coefs} is a label in the program section of the code, {\it not} the
data section).


\subsection{\it *(lk)}
\vspace{-0.1in}

{\it *(lk)} addressing is a syntactic oddity.  The star symbol generally
means that indirect addressing is being used (see below), but this is
actually direct addressing with a 16 bit data memory address encoded in
the instruction's last word.  The reason for the star here is that TI
{\it does} set the ``I'' bit in the opcode, usually denoting indirect
addressing, and this form can only be used when an {\it Smem} is called
for in the assembly syntax.  Other bits in the low byte of the first
instruction word tell the processor that the ``{\it *(lk)} exception''
is to be used, and to fetch the memory address in the next word (see the
MOD bits on page 5-10 of the {\it CPU Reference}).  You can easily
recognize this addressing mode in {\tt .lst} files because the low byte
of the first instruction word always equals {\tt F8 hex}.

Examples:

{\tt
\begin{tabular}{@{\hspace{0.4in}}l@{\hspace{0.3in}}l@{\hspace{0.3in}}l@{\hspace{0.2in}}l}
hold  & .word & 1   & ; reserve one word of storage and initialize to 1\\
count & .word & 0   & ; reserve one word of storage and initialize to 0\\
      & \dots\\

& ld   & *(count),b & ; sets B = 0 (assuming memory was not changed)\\
& st   & t,*(hold)  & ; sets (storage location at address hold) = T\\
\end{tabular}
}


\section{Indirect Addressing:  {\it Smem, Xmem, Ymem}}

{\it Indirect addressing} on the '54x always uses the auxiliary registers
AR0 through AR7, and comes in two basic flavors.  These are easily
recognized from the assembly language syntax descriptions as either
{\it Smem} or {\it Xmem/Ymem}.


\subsection{\it Smem}
\vspace{-0.1in}

In {\it Smem} indirect addressing, only one indirect address is used in the
instruction and a plethora of variations is possible (see the table on page
5-13 of the {\it CPU Reference}).  An asterisk is always used, which
signifies indirect addressing.  Any of the registers AR0--AR7 may be used,
with optional modifications:  automatic post-decrement by 1, pre- and
post-increment by 1, post-increment and post-decrement by $n$ ($n$ being
stored in AR0), and more, including a slew of options for circular
addressing (automatically implements circular buffers) and bit-reversed
addressing (useful for FFTs).


\subsection{\it Xmem/Ymem}
\vspace{-0.1in}

{\it Xmem/Ymem} indirect addressing is generally used in instructions
which need two different indirect addresses, although there are a few
instances where an {\it Xmem} by itself is specified in order to save
bits in the opcode for other options.
In {\it Xmem/Ymem} indirect addressing, Fewer bits are
used to encode the option modifiers in the opcode;  hence, fewer options
are available:  post-increment and post-decrement by 1, and post-increment
by AR0 with circular addressing.

Examples:

{\tt
\begin{tabular}{@{\hspace{0.4in}}l@{\hspace{0.3in}}l@{\hspace{0.2in}}l}
stl  & b,*ar6     & ; sets (contents of location pointed to by AR6) = low word of B\\
stl  & b,*ar6+0\% & ; sets (contents of location pointed to by AR6) = low word of B,\\
     &            & ; \hspace{0.3in} then increments AR6 with circular addressing\\
mar  & *+ar3(-6)  & ; decrements AR3 by 6 (increment by -6)\\
\end{tabular}
}

(Note:  The {\tt mar} (modify address register) instruction is unusual in
the sense that it takes an {\it Smem} operand but doesn't actually do
anything with the data pointed to by the ARx register.  Its purpose
is to perform any of the allowed register modifications discussed above
without having to do anything else.  This is often handy when you are using
an {\it Xmem/Ymem}-type instruction but need to do an ARx modification that
is only allowed with an {\it Smem}-type operand.)



\section{Summary}

The {\tt LD} instruction is illustrative of the many possible addressing
modes which can be selected with the proper choice of assembly language
syntax:

{\tt
\begin{tabular}{@{\hspace{0.2in}}l@{\hspace{0.3in}}l@{\hspace{0.2in}}l}
ld   & \#0,a       & ; immediate data:  sets A = 0\\
ld   & 0,a         & ; DP-referenced direct:  sets A = (contents of the address DP + 0)\\
ld   & mydata,a    & ; DP-referenced direct:  sets A = (contents of the address\\
     &             & ; \hspace{0.3in} DP + lower seven bits of mydata)\\
ld   & \#mydata,a  & ; immediate data:  sets A = 16 bit address mydata\\
ld   & *(mydata),a & ; *(lk) direct:  sets A = (contents of the 16 bit address mydata)\\
ld   & b,a         & ; accumulator:  sets A = B\\
ld   & *ar1+,a     & ; indirect:  sets A = (contents of address pointed to by AR1),\\
     &             & ; \hspace{0.3in} and afterwards increments AR1 by one\\
ldm  & ar2,a       & ; memory-mapped register:  sets A = AR2\\
\end{tabular}
}


\end{document}
