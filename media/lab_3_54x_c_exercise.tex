\section{Implementation}

As this is your first experience with the C environment, most of 
the programming for the assignment is to 
be done directly in assembly.  A C skeleton 
will provide access to input samples and a way to output samples.  
From the C skeleton, an assembly module for implementing the 
complete multirate system (for a single sample) is called.  In the 
assembly module, the downsampling and upsampling blocks are 
implemented by using a loop or counter to determine which samples 
to keep and when to insert zeros.

As there was a core file for working in the assembly environment
(Labs 0-2), there is a core file for the C environment 
(\verb+V:\ece320\54x\dspclib\core.asm+)
which handles the interrupts from the 
CODEC (A/D and D/A) and the serial port.  Here, we will describe the 
important aspects of the core code necessary to complete the assignment.  
The complete documentation on the core code developed for the C 
environment will be made available soon.  

\subsection{C Skeleton}
Let's examine the following C main program \verb+lab3main.c+ which 
calls assembly FIR filter functions \verb+init_filter+ and 
\verb+filter+.
\setlength{\baselineskip}{0.39cm}
\setlength{\parskip}{0.4cm}
\listinginput{1}{lab3main.c}
\setlength{\baselineskip}{0.5cm}
\setlength{\parskip}{0.5cm}

In the main program, an infinite loop operates over the input samples 
accessed by the pointer \verb+Rcvptr+ and writes the output samples 
via the pointer \verb+Xmitptr+.  In C, pointers may be used as array names 
so that \verb+Xmitptr[0]+ is the first word pointed to by \verb+Xmitptr+.  
The function \verb+WaitAudio+ is a assembly function in the core code 
which handles the CODEC interrupts.  It returns a block of \verb+BlockLen+
samples and writes \verb+BlockLen+ samples to each of the six 
channels.  As in the assembly core, the input samples are not in 
consecutive order.  The right and left inputs are offset from 
\verb+Rcvptr+ respectively by $4i$ and $4i+2$, 
$i=0,\ldots,\verb+BlockLen+-1$.  The six output channels are accessed 
consecutively as offsets from \verb+Xmitptr+.

\subsection{Assembly Functions}
Let's examine the calls to the assembly functions \verb+init_filter+ and 
\verb+filter+.  The assembly file containing these functions is 
\verb+v:\ece320\54x\dspclib\lab3filt.asm+

\setlength{\baselineskip}{0.39cm}
\setlength{\parskip}{0.4cm}
\listinginput{1}{lab3filt.asm}
\setlength{\baselineskip}{0.5cm}
\setlength{\parskip}{0.5cm}

The assembly file contains two main parts, the data section starting 
with \verb+.sect ``.data''+ and the program section starting with 
\verb+.sect ``.text''+.  Every function and variable accessed in C must be 
preceded by a single underscore \verb+_+ in assembly and 
a \verb+.global _name+ must be placed in the assembly file for linking.  
In this example, \verb+filter+ is an assembly function called from 
the C program with a label \verb+_filter+ in the text portion of the 
assembly file and a \verb+.global _filter+ declaration.  
In each assembly function, the macro \verb+ENTER_ASM+ is called upon 
entering and \verb+LEAVE_ASM+ is called upon exiting.  These 
macros are defined in \verb+v:\ece320\54x\dspclib\core.inc+.  The 
\verb+ENTER_ASM+ macro saves the status registers and 
\verb+AR1+,\verb+AR6+, and \verb+AR7+ when entering a function as 
required by the register use conventions.  The \verb+ENTER_ASM+ macro
also sets the status registers to the assembly conventions we 
have been using (i.e, \verb+FRCT+=1 for fractional arithmetic and 
\verb+CPL+=0 for \verb+DP+ referenced addressing).  
The \verb+LEAVE_ASM+ macro just restores the saved registers.

\subsection{Parameter Passing}
\label{sec:parameter passing}
The parameter passing convention between assembly and C is 
simple for single input, single output assembly functions.  From a 
C program, the input to an assembly program is in the low part of 
accumulator \verb+A+ with the output returned in the same place.  
In this example, the function \verb+filter+ takes the 
right input sample from \verb+A+ and returns a single output in 
\verb+A+ (note the left shift by $16$ to put the result in the low part of 
\verb+A+).  When more than one parameter is passed to an assembly function, 
the parameters are passed on the stack 
(see the core file description for more information).  We suggest that you 
avoid passing or returning more than one parameter.  Instead, use global 
memory addresses to pass in or return more than one parameter.  
Another alternative is to pass a pointer to the start of a buffer intended 
for passing and returning parameters.

\subsection{Registers Modified}

When entering and leaving an assembly function, the \verb+ENTER_ASM+ 
and \verb+LEAVE_ASM+ macros ensure that certain registers are saved 
and restored.  Since the C program may use any and all registers, the 
state of a register cannot be expected to remain the same between
calls to assembly function(s).  {\bf Therefore, any information that 
needs to be preserved across calls to an assembly function must be 
saved to memory !}  In this example, \verb+stateptr+ keeps track of the 
location of the current sample in the circular buffer \verb+firstate+.  
Why don't we need to keep track of the location of the coefficient pointer 
(\verb+AR2+ in this example) after every sample?

\subsection{Compiling and Linking}

A working program can be produced by compiling the C code and linking 
assembly modules and the core module.  The compiler translates C code 
to a relocatable assembly form.  The linker assigns physical addresses on 
the DSP to the relocatable data and code segments, resolves \verb+.global+ 
references and links runtime libraries.

The procedure for compiling C code and linking assembly modules has been 
automated for you in the batch file 
\verb+v:\ece320\54x\dsptools\C_ASM.bat+.  
Copy the files \verb+lab3main.c+, and 
\verb+lab3filt.asm+ from the \\
\verb+v:\ece320\54x\dspclib\+ directory 
into your own directory on the \verb+W:+ drive.  Using \matlab, 
write the coefficients you created in the prelab into 
a \verb+coef1.asm+ file.  Then, type 
\verb+c_asm lab3main lab3filt+ to produce a \verb+lab3main.out+ file 
to be loaded onto the DSP.  Load the output file onto the DSP as usual and 
check that is the FIR filter you designed.

\subsection{Cascade of FIR1 and FIR2}

Modify the \verb+lab3filt.asm+ assembly module to implement a 
cascade of filters FIR1 and FIR2.  Note that both \verb+_filter+ 
and \verb+_init_filter+ will need to be modified.  
Compile and link the new assembly module and confirm it has the 
frequency response which you expect from cascading FIR1 and FIR2.

\subsection{Complete Multirate System}

Once you have the cascaded system working, implement the multirate 
system composed of the three FIR filters by modifying the 
assembly modules in \verb+lab3filt.asm+.  In order to implement the 
sample rate converters, you will need to use a counter or a loop.  
{\bf The upsampling block and downsampling block are not implemented 
as seperate sections of code. }  Your counter or loop
will determine when the decimated rate processing is to occur
as well as when to insert zeros into FIR3 to implement the
zero-filling up-sampler. 

Some instructions that may be useful for implementing your
multirate structure are the \verb+addm+ (add to memory)
and \verb+bc+ (branch conditional) instructions. You may
also find the \verb+banz+ (branch on auxiliary register
not zero) instruction useful, depending on how you implement 
your code.  {\bf As the counter is state information that needs to 
be preserved between calls to \verb+filter+, the counter must be 
saved in memory.}

In order to experiment with multirate effects in your system, 
make the downsampling factor (D=U) a constant which can be changed 
easily in your code.  Is there a critical (D=U) associated with this 
system above which aliasing occurs?

It will be useful both for debugging and for experimentation to show 
the output of your system at various points in the block diagram.  
By modifying the C code in \verb+lab3main.c+ and the assembly 
modules in \verb+lab3filt.asm+, send the following sequences to the 
DSP output
\begin{itemize}
   \item{output of FIR1}
   \item{input to FIR2 (after downsampling)}
   \item{input to FIR3 (after upsampling)}
\end{itemize}
You will have to pass these samples to the main C program by storing 
them in memory locations as described in Section~\ref{sec:parameter
passing}.  Note that the input to FIR2 is at the downsampled 
rate.

\paragraph{Grading and Oral Quiz}
For the quiz, you should be prepared to change the decimation
rate upon request, and explain the effects of changing
the decimation rate on the system's output.

As usual, your grade will be split up into three sections:
\begin{itemize}
\item 1 point: Prelab
\item 4 points: Code (Code which is complete and working at the 
        beginning of the lab period gets full credit.)
\item 5 points: Oral Quiz
\end{itemize}
The oral quiz may cover various problems in multirate sampling theory,
as well as the operation of your code itself and details about the
instructions you've used in your code. Be prepared to explain, in
detail, the operation of all of your code, even if your lab partner
wrote it!  You may also be asked to make changes to your code and
to predict, and explain, the effects of these changes.

\paragraph{Extra Credit: 1 point}
One of the main benefits of multirate systems is efficiency.  Because of 
downsampling, the output of FIR1 is used only one of $D$ times.  Make 
your assembly module more efficient by using this fact.  

Similarly, at the input of FIR3, $D-1$ of every $D$ samples is zero.  
So, for a fixed downsampling factor $D$, it is possible to make use of 
this fact to create $D$ different filters (each a subset of the 
coefficients of FIR3) to be used at the $D$ time instances.  This 
technique is referred to as polyphase filtering and can be found 
in most modern DSP textbooks.  These filters are more efficient as 
the sum of the lengths of the filters is equal to the length of FIR3.  
Apply this fact for $D=4$.
