
%
%
% Module: evm_memory_maps_54x_tutorial
%
% Author: Daniel Sachs
%
%

As you've seen in previous labs, the TI C549 DSP has two seperate
memory spaces, called Program and Data. The data space is 64k long,
as we'd expect given the 16-bit addressing registers. The program
space, however, can address up to 8192k words using the DSP's
"extended addressing" modes. These modes include far calls that
reset the full 23-bit program counter and accumulator-addressed
transfer instructions. A small amount of space at the start of
data memory (addresses from 0000-007F) are reserved for memory-
mapped registers, scratchpad memory, and onboard peripherals, and a
small amount of space (FF80-FFFF) at the end of the first 64k page of
program memory is reserved for interrupt vectors.

\subsection{Internal vs. External Memory}

One of the issues we have not yet addressed is the difference between
internal
memory (located on the DSP chip itself) and external memory (located on the
evaluation board). Although the external memory is much larger, the internal
memory has several advantages: it's faster (zero wait states), and can be
accessed several times in one cycle. External memory, on the other hand,
can only be accessed one word at a time, and each access takes an extra
two cycles due to the wait states required to match the 15ns memory and
address-decode GALs to the 100MHz DSP.

For your project, you want to use the internal memory to hold your code,
filter coefficients, and small buffers you use to generate your audio
effects. External memory should be used for large buffers that you
only access a few times per sample, like the delay buffer described
in this lab.

\subsection{TMS320C549 DSP EVM Memory Maps}

The EVM's data address space is addressed fully by the 16-bit addressing
(AR) registers and address extension words. Program memory,
however, is split up into 64K (16-bit) pages by the hardware. Accessing
code or data stored outside the current page requires the use of special
instructions. The far call instructions are required to jump to code in
another 64k page; since we don't expect you to generate 64k words of
code for your lab project, you shouldn't need this. However, it is useful
to use the extra memory available in the program RAM space for storage
of large amounts of data; this data can be used in many ways, including
the delay buffer described in this handout.

\begin{figure}[htb]\centerline  {
\epsfxsize=.6\textwidth\
\epsffile{ram.eps}          }
\caption{DSP EVM memory maps}
\label{fig: memmap}
\end{figure}

The TMS320C549 DSP we are using in ECE 320 allows its internal memory to
be switched into the program memory space using the OVLY bit in the PMST,
a processor state control register. If the internal RAM is switched in,
it appears in both the data and memory address space at locations 0080h to
07FFFh. This means that if the internal memory is switched in, anything
written into data memory below 07FFFh will overwrite a program stored in
the same location. In addition, copies of the internal memory also appear
in the extended program address space, occupying locations 0080-7FFF of
each page. Therefore, with the overlay mode set, any addresses to program
memory locations in the form of xx0000-xx7FFF reference internal memory.

With the OVLY bit set at zero, internal memory is disabled in program space
and the program memory map includes only external memory. In this mode, the
entire 192k words of external program RAM is accessable, although
several wait states will be required for accessing each item of memory.
In the overlay mode, only addresses in the ranges of 08000h to 0FF80h,
1800h to 1FFFFh, and 28000h to 2FFFFh are available for your data; the
remaining addresses are unmapped or map from the on-chip RAM.

In addition, ensure that your code does not overwrite program memory between
0FF80h and 0FFFFh; this area is reserved for the interrupt vector table.

