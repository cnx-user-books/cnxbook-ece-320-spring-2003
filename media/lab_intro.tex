\documentclass[11pt]{handout}

\renewcommand{\coursetitle}{ECE 320}
\renewcommand{\handouttitle}{Course Overview Handout}
\renewcommand{\semestertitle}{Spring 2002}

\begin{document}
\setlength{\baselineskip}{0.5cm}
\setlength{\parskip}{0.5cm}

\makeboxtitle
\vspace{0.3cm}

\begin{tabbing}
{\bf Lecturer:} \= Professor Yoram Bresler \hspace{1.2in} \= 
{\bf \hspace{1.0in}} \= \\
\> 112 CSL   \> \> \\
\> 244-9660 \> \> \\
\> ybresler@uiuc.edu \> \> \\
 \\
{\bf T.A.s:} 
\> Matt Berry \> \> Dima Moussa \\
\> mjberry@uiuc.edu \> \> dmoussa@uiuc.edu \\
 \\
\> Mike Russell \> \> Brian Wade \\ 
\> merussel@uiuc.edu \> \> bwade@uiuc.edu \\
 \\ 
 \\
{\bf Lecture:} 
\> Monday 2:00, 269 Everitt Lab \> {\bf Lab:} \>
251 Everitt Lab,  \hspace{.2in} 244-1360 \\
\> \> \> Mon., Fri.: 3--5 \\
\> \> \> Tue., Wed., Thurs.: 2--4 \\
 \\
\end{tabbing}

\paragraph{Web page:} http://www.ews.uiuc.edu/\~{}ece320 

\paragraph{Recommended Text:}
Texas Instruments,
{\em TMS320C54X DSP Reference Set Volume 1: CPU and Peripherals, 
Volume 2: Mnemonic Instruction Set}, and {\em Volume 4: Applications Guide}.
These documents are available in PDF and PS formats on the course web page 
in the ``handouts'' section and are also available in hard copy in the lab.  
It is not necessary to purchase any texts for this course, and we ask that 
you {\bf do not} print the manuals on the lab printer.

\paragraph{Office hours:} Office hours are held in the lab.  Times will 
be announced later.  

\section{Introduction}

The intent of this course is to familiarize students
with the fundamentals of 
operating and analyzing real-time digital signal processing
(DSP) systems including the
theory required, the hardware used to sample and process the signals, and
the software environment used to control the system.
The theory is
primarily those DSP concepts covered in ECE 310 including sampling, 
convolution,
filtering, filter-design, modulation, and multirate processing
(interpolation and decimation).  The DSP hardware consists of
an analog-to-digital (A/D) and digital-to-analog (D/A) converters 
and a TI-549  DSP to perform the processing.

%\newpage

\section{Schedule}

The first half of the course consists of semi-self-paced labs
in which you will learn the specifics of the TI-54X DSP,  its assembly 
language, and its compilers.
For the second half of the semester, you are to select and
complete a real-time DSP-related project.  Note in the schedule 
below that, except for the first week of class, a ``lab week'' starts on a 
Tuesday and ends on the following Monday.

\begin{tabbing}
Dates \hspace{.75in} \= Lecture \hspace{1.3in} \= Lab \hspace{2.0in}
\= Requirements\\
 \\
Jan. 14 -- 21 \> Introduction \> Lab 0: Lab Orientation\\
Jan. 22 -- 28 \> TI Assembly Language \> Lab 1: FIR Filtering \>
{\em Prelab 1} \\
Jan. 29 -- Feb. 4 \> FIR / IIR Filters \> Lab 2: IIR Filtering \>
{\em Lab 1 quiz, Prelab 2} \\
Feb. 5 -- 11 \> Compilers \> Lab 3: Multirate FIR Filtering \>
{\em Lab 2 quiz, Prelab 3} \\
Feb. 12 -- 18\> FFTs \> Lab 4: Spectrum Analyzer\>
{\em Lab 3 quiz, Prelab 4}  \\
Feb. 19 -- 25 \> Digital Communications \> 
Lab 5: Digital Communications \>
{\em Lab 4 quiz}  \\
Feb. 26 -- Mar. 4 \> Special Topics \> continue Lab 5 \> {\em Prelab 5} \\

Mar. 5 -- 11 \> Special Topics \> Project Lab 1\>
{\em Lab 5 quiz} \\
\> \> \>{\em Project abstract}\\
Mar. 12 -- 25 \> Special Topics \> Project Lab 2 \> {\em Project quiz 1}\\
Mar. 26 -- Apr. 1 \> Special Topics \> Project Topic Feedback \>
{\em Project quiz 2} \\
Apr. 2 -- 8 \> Special Topics \> Design Review Presentations \>
{\em Design review slides} \\
Apr. 9 -- 15 \> Special Topics \> Project \> {\em Pass design review}\\
Apr. 16 -- 22 \> Special Topics \> Project \\
Apr. 23 -- 29 \>  Special Topics \> Project \\
Apr. 30 -- May 1 \> \> \> {\em Project demonstrations}\\
May 1 \> \> \> {\em Project reports}\\
\end{tabbing}

\section{Grading}

The structured laboratory segment will count for
50\% of the total grade, based on completion of, and oral examination
over, the weekly exercises with each student quizzed individually.
Labs are worth 10 points, usually with 1 point for prelab completion,
4 points for working code, and the remaining 5 points
for quiz performance.
We emphasize that grading in this class is based heavily
on your demonstration of course material, rather than
exams or submitted assembly code.

The project will count for 50\% of the total
grade, with 20\% of the total grade dependent on technical work on
and oral demonstration of the
project, and 10\% of the grade dependent on the completeness and
quality of the design review, 10\% for the final report
and 5\% each on project labs. 

It is expected that each student will attend and participate in scheduled 
class and laboratory meetings and report on progress, or will make 
{\bf prior} other arrangements with the instructor or T.A.s.  The final grade 
may be penalized if this does not occur.  

\paragraph{Assignments:}  All graded assignments, including prelab exercises, 
DSP code, and final project materials, must be submitted to receive a grade 
in the course.  All assignments other than the final report and presentation 
are due at the start of your scheduled laboratory meeting.  You must have your 
code complete before the class begins on the day the lab will be quizzed.  
We reserve the right to consider code late if it is not complete before the 
start of the lab session in which it is due.  A late penalty of 50\% will be 
assessed for assignments less than a week late, and no credit will be given 
for assignments more than one week late.  In addition to these policies, the 
final project abstract must be submitted and approved before project labs or 
project work are accepted for grading.  Similarly, the design review must be 
passed before demonstration, report submission, or grading of the final project 
is allowed.

\paragraph{Quizzes:} All lab quizzes must be taken during the lab on the day 
the lab is due.  You must take the quiz during your assigned laboratory 
period even if your lab assignment is not complete.  Any quiz not taken during 
its assigned laboratory period will be assigned a zero grade unless other 
arrangements are made in advance with your section TAs or Professor Bresler.  
Exceptions to this policy will be granted only for excuses recognized 
by the College of Engineering.

\section{Lab Access}

We will meet in the lab for two hours during the week at the
assigned times.  In addition, you will be able to access
the lab at any time during the semester using a keycard.

Completion of your lab assignments will generally 
entail additional lab time outside of the scheduled hours.
Basic rules of courtesy must be followed when in the lab.  
Please do not remove any lab equipment, books or manuals from 
the lab at any time.  Do not bring food or drink into the lab.
If you would like to listen to music as you work, please use 
headphones.  

%This includes: 
%No Messy Food or Drink! especially meat and poultry!
%no fighting !
%No destroying or damaging equipment!
%All hardware must stay in the lab!

\paragraph{Keycard Access:}
Off-hour lab time is available by way of keycard access into
Everitt Lab as well as the DSP Lab through room 251.  Those
students who currently have keycards should already have
their cards activated for the DSP lab.  If you are registered
for the class and do not have a keycard, please request one  
in room 153 E.L..

%\section{Projects}
%
%As mentioned, the goal of the laboratory exercises is
%to provide you  ...

\end{document}

