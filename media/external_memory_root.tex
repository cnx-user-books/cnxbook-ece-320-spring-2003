\documentclass[11pt]{handout}
\usepackage{moreverb}
\usepackage{epsf}
\usepackage{epic}
\usepackage{eepic}

\renewcommand{\coursetitle}{ECE 320}
\renewcommand{\handouttitle}{Audio Effects: Delay}
\renewcommand{\handoutauthor}{Daniel Sachs}
\renewcommand{\semestertitle}{Fall 1999}

\newcommand{\bea}{\begin{eqnarray}}
\newcommand{\eea}{\end{eqnarray}}

\setlength{\parindent}{5mm}
\begin{document}

\setlength{\baselineskip}{0.5cm}
\setlength{\parskip}{0.5cm}

\makeboxtitle
\vspace{0.3cm}

\section{Introduction}

In this project lab, we will introduce the DSP memory
map and describe how to store a large amount of data on the EVM.
We will then use this to implement a long audio delay and an
audio echo.

The TI DSP evaluation boards we are using in this class have a large
amount of memory - 32k words internal to the DSP itself and
another 256k words on the EVM board. During the lab portion 
of this course, we used only
the board-mounted memory, and didn't really consider how much
we were using or the differences between internal and external RAM.
However, in order to generate large delays - which are often required in 
audio processing - we need to use a significant portion of the
memory on the DSP board. And in order to achieve the DSP's peak 
performance, we need to use the on-chip memory to reduce the time
the DSP spends waiting for data.

In this lab, you will explore the memory map of the DSP board,
and learn how to access all of the memory on the board. You will
then apply this knowledge to build a very long delay line. Finally,
you will modify the delay line to delay the input by an arbitrary
number of points, then feed back the delayed signal into the input.
This will result in a simple echo effect.

\section{EVM Memory Maps}


%
%
% Module: evm_memory_maps_54x_tutorial
%
% Author: Daniel Sachs
%
%

As you've seen in previous labs, the TI C549 DSP has two seperate
memory spaces, called Program and Data. The data space is 64k long,
as we'd expect given the 16-bit addressing registers. The program
space, however, can address up to 8192k words using the DSP's
"extended addressing" modes. These modes include far calls that
reset the full 23-bit program counter and accumulator-addressed
transfer instructions. A small amount of space at the start of
data memory (addresses from 0000-007F) are reserved for memory-
mapped registers, scratchpad memory, and onboard peripherals, and a
small amount of space (FF80-FFFF) at the end of the first 64k page of
program memory is reserved for interrupt vectors.

\subsection{Internal vs. External Memory}

One of the issues we have not yet addressed is the difference between
internal
memory (located on the DSP chip itself) and external memory (located on the
evaluation board). Although the external memory is much larger, the internal
memory has several advantages: it's faster (zero wait states), and can be
accessed several times in one cycle. External memory, on the other hand,
can only be accessed one word at a time, and each access takes an extra
two cycles due to the wait states required to match the 15ns memory and
address-decode GALs to the 100MHz DSP.

For your project, you want to use the internal memory to hold your code,
filter coefficients, and small buffers you use to generate your audio
effects. External memory should be used for large buffers that you
only access a few times per sample, like the delay buffer described
in this lab.

\subsection{TMS320C549 DSP EVM Memory Maps}

The EVM's data address space is addressed fully by the 16-bit addressing
(AR) registers and address extension words. Program memory,
however, is split up into 64K (16-bit) pages by the hardware. Accessing
code or data stored outside the current page requires the use of special
instructions. The far call instructions are required to jump to code in
another 64k page; since we don't expect you to generate 64k words of
code for your lab project, you shouldn't need this. However, it is useful
to use the extra memory available in the program RAM space for storage
of large amounts of data; this data can be used in many ways, including
the delay buffer described in this handout.

\begin{figure}[htb]\centerline  {
\epsfxsize=.6\textwidth\
\epsffile{ram.eps}          }
\caption{DSP EVM memory maps}
\label{fig: memmap}
\end{figure}

The TMS320C549 DSP we are using in ECE 320 allows its internal memory to
be switched into the program memory space using the OVLY bit in the PMST,
a processor state control register. If the internal RAM is switched in,
it appears in both the data and memory address space at locations 0080h to
07FFFh. This means that if the internal memory is switched in, anything
written into data memory below 07FFFh will overwrite a program stored in
the same location. In addition, copies of the internal memory also appear
in the extended program address space, occupying locations 0080-7FFF of
each page. Therefore, with the overlay mode set, any addresses to program
memory locations in the form of xx0000-xx7FFF reference internal memory.

With the OVLY bit set at zero, internal memory is disabled in program space
and the program memory map includes only external memory. In this mode, the
entire 192k words of external program RAM is accessable, although
several wait states will be required for accessing each item of memory.
In the overlay mode, only addresses in the ranges of 08000h to 0FF80h,
1800h to 1FFFFh, and 28000h to 2FFFFh are available for your data; the
remaining addresses are unmapped or map from the on-chip RAM.

In addition, ensure that your code does not overwrite program memory between
0FF80h and 0FFFFh; this area is reserved for the interrupt vector table.



To make this lab simpler, you will
may leave the overlay mode off and disable internal program memory for
this lab. In Lab II, we will address switching the overlay mode on
and off "on the fly" to achieve better performance.

Note that the \verb+thru.asm+ file you have been given disables the
overlay mode, so you won't need to worry about switching out of it
for this lab.

\section{Accessing Extended Program RAM}


%
%
% Module: accessing_extended_program_ram_54x_tutorial
%
% Author: Daniel Sachs
%
%

The TMS320C549 DSP includes accumulator-addressed program memory read
and write instructions: \verb+readprog+ and \verb+writprog+. These allow the
processor to read or write from data memory to an arbitrary location
in program memory. Data memory is addressed using a standard memory
reference; program memory is addressed using the contents of the
accumulator A.

For instance, the following code fragment loads the value contained
in memory location 023456h into the location 0064h in data memory
using the \verb+readprog+ opcode:

\setlength{\baselineskip}{0.4cm}
\listinginput{1}{readprog_example.asm}
\setlength{\baselineskip}{0.5cm}

The \verb+writprog+ opcode can be used similary to write into extended program
RAM:

\setlength{\baselineskip}{0.4cm}
\listinginput{1}{writprog_example.asm}
\setlength{\baselineskip}{0.5cm}

Note that Code Composer will not show you or allow you to change the
contents of the extended program
RAM on the memory-dump or disassembly screen, though you can or change it
indirectly by watching the effects of the \verb+readprog+ and \verb+writprog+
opcodes in data memory.



\section{Delay and Echo Implementation}


%
%
% Module: external_memory_processor_exercise
%
% Author: Daniel Sachs
%
%

You will be implementing three audio effects for this lab: a 2-second delay,
a variable delay, and an feedback echo.

\subsection{Delay Implementation}

For this project lab, you will implement a 131,072 sample
delay using external memory. Use memory locations 010000h - 02ffffh in
extended program RAM to do this; you'll probably also want to use the
\verb+dld+ and \verb+dst+ opcodes to store and retrieve the 32-bit
addresses from the accumulators. Note that these two operations store
the words in memory in big endian order, with the high-order word first.

Use the \verb+reada+ and \verb+writa+
instructions to read and write the extended program memory.
Since the \verb+thru.asm+ file you were given in Lab 1 clears the
OVLY bit, you can assume that you can read or write from all of the
extended program memory without overwriting you program code or
data area.

Remember that arithmetic operations that act on the accumulators, such
as the \verb+add+ instruction, operate the complete 32 or 40 bit wide
value. Also keep in mind that since 131,072 is a power of two, you can
use masking (via the \verb+and+ instruction) to implement the circular
buffer easily.  This delay will be easy to verify on the oscilloscope.

\subsection{Variable delay implementation}

Once you have your delay code working with a 131,072 sample delay,
change it so that the delay can be easily changed to any arbitrary
length between zero (or one) and 131,072 samples by changing the value
stored in one double-word pair in memory. You should keep the buffer
length as 131,072 and change only your addressing of the sample being
read back; it is more difficult to change the buffer size to a length
that is not a power of two.

Verify that your code works as expected by timing the delay from
input to output and ensuring that it is approximately the
right length.

\subsection{Feedback echo implementation}

Last, chnage your code so that the value taken from the end of
the variable delay above is multiplied by a gain factor and
then added back into the input, and the result is both saved
into the delay line and sent out to the digital-to-analog converters.
(It may be necessary to multiply
the input by a gain as well to prevent overflow.) This will make
a one-tap feedback echo, an simple audio effect that sounds remarkably
good.

To test this, hook up the DSP EVM input to the CD player on the
computer, and the output to one of the pairs of speakers in the
lab. This can be done using the RCA to 1/8 inch (headphone jack)
converter plugs available in the lab.
Verify that the echo can be heard multiple times, and that the
spacing between echos matches the delay length you've chosen.


\end{document} 
